% Created 2010-07-04 Sun 15:02
\documentclass[pre,preprint]{revtex4-1}


\title{Draft of diameter paper}
%\author{Don Blair}
\date{04 July 2010}

\begin{document}



\title{ \emph{Physicae Auscultationes}}
\author{D. W. Blair}
\date{September 2010}
\maketitle

\section{Melting A}
\label{sec-1}
\subsection{Background re: melting}
\label{sec-1.1}
\subsubsection{Theories of melting, 3D, 2D, bulk}
\label{sec-1.1.1}
\paragraph{3D crystallites w/ stable surfaces melt from within via Born melting}
\label{sec-1.1.1.1}

In this case, melting can be viewed as nucleation and growth of fluid phase within the solid.
\subparagraph{or yet another structure.}
\label{sec-1.1.1.1.1}
\begin{itemize}

\item or even another\\
\label{sec-1.1.1.1.1.1}%
\end{itemize} % ends low level
\paragraph{2D large crystallites melt by two-step process via hexatic phase}
\label{sec-1.1.1.2}
\paragraph{2D finite crystallites melt from perimeter}
\label{sec-1.1.1.3}
\subparagraph{if melt from perimeter, dN/dt goes as $N^{1/2}$}
\label{sec-1.1.1.3.1}
\subsubsection{Expectations for 2D finite crystallites}
\label{sec-1.1.2}
\subsection{Experiment of Savage et. al}
\label{sec-1.2}
\subsubsection{Setup}
\label{sec-1.2.1}
\subsubsection{Tunable Depletion potential}
\label{sec-1.2.2}
\subsubsection{Results}
\label{sec-1.2.3}
\paragraph{N vs. t}
\label{sec-1.2.3.1}
\paragraph{$< psi6 >^2$ vs. N}
\label{sec-1.2.3.2}
\paragraph{$C_6$ vs. N, by layer}
\label{sec-1.2.3.3}
\paragraph{No dependence of fast-melting feature on initial cluster size or melting rate}
\label{sec-1.2.3.4}
\subsection{Simulations}
\label{sec-1.3}
\subsubsection{Motivation}
\label{sec-1.3.1}
\subsubsection{GROMACS System}
\label{sec-1.3.2}
\subsubsection{Brownian dynamics}
\label{sec-1.3.3}
\subsubsection{Simulated Depletion Potential}
\label{sec-1.3.4}
\subsubsection{Simulated Lennard-Jones Potential}
\label{sec-1.3.5}
\subsubsection{Results}
\label{sec-1.3.6}
\paragraph{N vs. t}
\label{sec-1.3.6.1}
\paragraph{$< psi6 >^2$ vs. N}
\label{sec-1.3.6.2}
\paragraph{$C_6$ vs. N, by layer}
\label{sec-1.3.6.3}
\paragraph{mean-square fluctuations in bond lengths}
\label{sec-1.3.6.4}
\paragraph{N vs. t for Lennard-Jones potential}
\label{sec-1.3.6.5}
\paragraph{Phase diagram showing lack of fluid phase with short-range potential}
\label{sec-1.3.6.6}
\subsubsection{Discussion}
\label{sec-1.3.7}
\section{Melting B}
\label{sec-2}
\subsection{Background}
\label{sec-2.1}
\subsubsection{Hypothesis: thermally-activated defects enhance melting rate in short-range 2D system}
\label{sec-2.1.1}
\subsection{Simulation Methods}
\label{sec-2.2}
\subsubsection{Gromacs system}
\label{sec-2.2.1}

Here's a good test. \cite{Deng2009}
\subsubsection{Brownian Dynamics}
\label{sec-2.2.2}
\subsubsection{Characteristics of Simulated Depletion Potential}
\label{sec-2.2.3}
\subsubsection{Initial configurations}
\label{sec-2.2.4}
\subsection{Results}
\label{sec-2.3}
\subsubsection{N vs t}
\label{sec-2.3.1}
\subsubsection{Order vs. N}
\label{sec-2.3.2}
\subsubsection{Breakdown by layers}
\label{sec-2.3.3}
\subsection{Conclusions}
\label{sec-2.4}
\section{Diameter of Random Clusters}
\label{sec-3}
\subsection{Background}
\label{sec-3.1}
\subsection{Simulations}
\label{sec-3.2}
\subsection{Results}
\label{sec-3.3}
\section{Phase Transitions in Computational Complexity}
\label{sec-4}
\subsection{Background}
\label{sec-4.1}
\subsubsection{Constraint Satisfaction Problems (CSP)}
\label{sec-4.1.1}
\paragraph{Examples}
\label{sec-4.1.1.1}
\subparagraph{kSAT}
\label{sec-4.1.1.1.1}
\subparagraph{Graph-coloring}
\label{sec-4.1.1.1.2}
\subparagraph{Spin models}
\label{sec-4.1.1.1.3}
\subparagraph{error-correcting codes}
\label{sec-4.1.1.1.4}
\paragraph{Observation of threshold behavior in CSP}
\label{sec-4.1.1.2}
\paragraph{Difficulties in tackling phase behavior of CSP}
\label{sec-4.1.1.3}
\subsubsection{Proposal: study complexity of percolation model}
\label{sec-4.1.2}
\subsection{Percolation}
\label{sec-4.2}
\subsubsection{The Model}
\label{sec-4.2.1}
\subsubsection{Background / applications}
\label{sec-4.2.2}
\subsection{PRAM}
\label{sec-4.3}
\subsubsection{Applications in comp sci}
\label{sec-4.3.1}
\subsubsection{PRIORITY CRCW}
\label{sec-4.3.2}
\subsection{Parallel Algorithm for Percolation}
\label{sec-4.4}
\subsection{Results}
\label{sec-4.5}
\subsubsection{D$_2$ vs. p for several system sizes L}
\label{sec-4.5.1}
\subsubsection{log(D$_2$) vs. log(L)}
\label{sec-4.5.2}
\subsubsection{Distribution of cluster sizes}
\label{sec-4.5.3}
\paragraph{logarithmic or power law? (power law --> algorithm will often fail)}
\label{sec-4.5.3.1}
\section{Bibliography}
\label{sec-5}

\bibliographystyle{plain}
\bibliography{/home/dwblair/Dropbox/dwbdocs/physics/writing/bibfiles/combo}

 

\end{document}