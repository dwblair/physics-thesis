% Created 2010-07-01 Thu 12:02
\documentclass{umthesis}


\title{Draft of diameter paper}
%\author{Don Blair}
\date{01 July 2010}

\begin{document}



\title{ \emph{Physicae Auscultationes}}
\author{D. W. Blair}
\date{September 2010}

\copyrightyear{2010}
\bachelors{B.Sc.}{University of Massachusetts Amherst}
\masters{M.Sc.}{University of Massachusetts Amherst}
%\secondmasters{M.Ed.}{Antioch College}
%\priordoctorate{M.D.}{University of Never-never-land}
% \committeechair{B. B. Bahh}
\cochairs{B. B. Bahh}{I. M. A. Wolf}
\firstreader{Little Bo Peep}
\secondreader{R. U. Sheepish}
\thirdreader{Bill Shepherd}
\fourthreader{Mary Lamb}   % Optional
%\fifthreader{}            % Optional
%\sixthreader{}            % Optional
\departmentchair{Don Candela}
\departmentname{Physics Department}


\frontmatter
\maketitle
%\copyrightpage
%\signaturepage


%\begin{dedication}
%  \begin{center}
%    \emph{To those little lost sheep.}
%  \end{center}
%\end{dedication}

%\chapter{Acknowledgments}
%Acknowledgements.

%\begin{abstract} 
%Test of abstract
%\end{abstract}

\tableofcontents                % Table of contents
%\listoftables                   % List of Tables -- ADD BACK IN
%\listoffigures                  % List of Figures -- ADD BACK IN
\mainmatter

%\unnumberedchapter{Introduction}
%Why on earth do I want to study sheep anyway?


\chapter{Melting: Part A (science paper)}
\label{sec-1}
\section{Background}
\label{sec-1.1}
\subsection{theory of 2-D melting}
\label{sec-1.1.1}
\begin{itemize}

\item bulk: hexatic, two-stage\\
\label{sec-1.1.1.1}%
\item for finite crystallites: melting dominated by surface\\
\label{sec-1.1.1.2}%
\item range of potential as issue\\
\label{sec-1.1.1.3}%
\item imaging small crystallites difficult before savage's technique\\
\label{sec-1.1.1.4}%
\end{itemize} % ends low level
\section{Experiments by Savage et. al}
\label{sec-1.2}
\section{Depletion potential}
\label{sec-1.3}
\subsection{range \~{} 10 \% of particle diameter}
\label{sec-1.3.1}
\subsection{observation: sublimation at steady rate until characteristic size, then enhanced melting}
\label{sec-1.3.2}
\subsection{Figure: N vs t (Fig 2. from savage et. al)}
\label{sec-1.3.3}
\subsection{Figure: Q$_6$ vs t (Fig 3. from savage et. al)}
\label{sec-1.3.4}
\section{Simulations}
\label{sec-1.4}
\section{Motivation}
\label{sec-1.5}
\subsection{confirm that odd hydrodynamics didn't play a role}
\label{sec-1.5.1}
\subsection{explore role of range of potential on melting}
\label{sec-1.5.2}
\section{Simulation algorithm / details}
\label{sec-1.6}
\subsection{brownian dynamics simulation}
\label{sec-1.6.1}
\begin{itemize}

\item theory\\
\label{sec-1.6.1.1}%
\item algorithm\\
\label{sec-1.6.1.2}%
\item form of the interaction potential used
\label{sec-1.6.1.3}%
\begin{itemize}

\item A-O depletion model\\
\label{sec-1.6.1.3.1}%
\item 'Blairium' -- A-O, but avoid infinite Brownian dynamics force\\
\label{sec-1.6.1.3.2}%
\end{itemize} % ends low level
\end{itemize} % ends low level
\subsection{phase diagram exploration}
\label{sec-1.6.2}
\section{Results}
\label{sec-1.7}
\subsection{short-range potential (\~{}10\%)}
\label{sec-1.7.1}
\begin{itemize}

\item N vs. t\\
\label{sec-1.7.1.1}%
\item Q$_6$ vs. t\\
\label{sec-1.7.1.2}%
\item Q$_6$ vs. N\\
\label{sec-1.7.1.3}%
\end{itemize} % ends low level
\subsection{longer-range potential (\~{}80\%)}
\label{sec-1.7.2}
\section{Discussion}
\label{sec-1.8}
\section{Future work}
\label{sec-1.9}
\subsection{3D}
\label{sec-1.9.1}
\subsection{curved surfaces}
\label{sec-1.9.2}
\subsection{non-spherical molecules}
\label{sec-1.9.3}
\chapter{Melting: Part B (moumita)}
\label{sec-2}
\section{Background}
\label{sec-2.1}
\subsection{Reference to experimental work and theory work in melting A chapter}
\label{sec-2.1.1}
\subsection{Theory: range of potential controls brittle/ductile transition}
\label{sec-2.1.2}
\begin{itemize}

\item brittle / ductile theory\\
\label{sec-2.1.2.1}%
\end{itemize} % ends low level
\subsection{When crystallites are sufficiently brittle, melting is mediated by defects}
\label{sec-2.1.3}
\begin{itemize}

\item alternative melting models\\
\label{sec-2.1.3.1}%
\end{itemize} % ends low level
\section{Theory}
\label{sec-2.2}
\subsection{Determine energy cost, E, of creating a disclination on flat 2D Membrane}
\label{sec-2.2.1}
\subsection{For thermally-activated disclinations, K$_B$ T \~{} E}
\label{sec-2.2.2}
\subsection{Disclinations create internal stresses: relieved by cracking}
\label{sec-2.2.3}
\subsection{potential energy penalty, V, of crack in 2D sheet}
\label{sec-2.2.4}
\subsection{minimize V to find critical crack length, l$_c$(Y,gamma)}
\label{sec-2.2.5}
\subsection{estimate Y, gamma for simulations}
\label{sec-2.2.6}
\subsection{use these to find a, critical average interparticle separation}
\label{sec-2.2.7}
\subsection{this allows us to find a critical potential range, a}
\label{sec-2.2.8}
\section{Simluations}
\label{sec-2.3}
\subsection{Brownian dynamics background (refer to previous)}
\label{sec-2.3.1}
\subsection{My code (include in thesis) vs. Gromacs}
\label{sec-2.3.2}
\subsection{New form of the interparticle potential}
\label{sec-2.3.3}
\begin{itemize}

\item plot for short-range, longer-range\\
\label{sec-2.3.3.1}%
\end{itemize} % ends low level
\subsection{Phase behavior / melting temperature}
\label{sec-2.3.4}
\begin{itemize}

\item brittle\\
\label{sec-2.3.4.1}%
\item ductile\\
\label{sec-2.3.4.2}%
\end{itemize} % ends low level
\subsection{Results}
\label{sec-2.3.5}
\begin{itemize}

\item For Brittle and Ductile cases:
\label{sec-2.3.5.1}%
\begin{itemize}

\item N vs. t\\
\label{sec-2.3.5.1.1}%
\item Q$_6$ vs. (N-N*)\\
\label{sec-2.3.5.1.2}%
\item Q$_6$ vs. (N-N*)\\
\label{sec-2.3.5.1.3}%
\item Ave. topological charge vs. (N-N*)\\
\label{sec-2.3.5.1.4}%
\end{itemize} % ends low level

\item Alternative theories: e.g. Lacoste\\
\label{sec-2.4.1}%
\end{itemize} % ends low level
\section{Discussion}
\label{sec-2.4}
\section{Code}
\label{sec-2.5}
\chapter{Diameter of Random Clusters}
\label{sec-3}
\section{Background}
\label{sec-3.1}
\subsection{Applications and physical realizations of the potts model}
\label{sec-3.1.1}
\subsection{Interesting properties of potts model clusters}
\label{sec-3.1.2}
\begin{itemize}

\item mass\\
\label{sec-3.1.2.1}%
\item perimeter\\
\label{sec-3.1.2.2}%
\item chemical distance
\label{sec-3.1.2.3}%
\begin{itemize}

\item literature review
\label{sec-3.1.2.3.1}%
\begin{itemize}

\item applications\\
\label{sec-3.1.2.3.1.1}%
\end{itemize} % ends low level

\item current understanding\\
\label{sec-3.1.2.3.2}%
\item no established relationship to other scaling exponents\\
\label{sec-3.1.2.3.3}%
\end{itemize} % ends low level

\item diameter
\label{sec-3.1.2.4}%
\begin{itemize}

\item graph theoretic definition\\
\label{sec-3.1.2.4.1}%
\item applications
\label{sec-3.1.2.4.2}%
\begin{itemize}

\item relevant to efficiency of simulations\\
\label{sec-3.1.2.4.2.1}%
\item communication on a potts network\\
\label{sec-3.1.2.4.2.2}%
\end{itemize} % ends low level

\item mean field expectations\\
\label{sec-3.1.2.4.3}%
\end{itemize} % ends low level
\end{itemize} % ends low level
\subsection{Review of potts model}
\label{sec-3.1.3}
\begin{itemize}

\item overview\\
\label{sec-3.1.3.1}%
\item phase behavior for q=1,2,3,4, D=1,2,3,4,infinite\\
\label{sec-3.1.3.2}%
\end{itemize} % ends low level
\section{Simulations}
\label{sec-3.2}
\subsection{swendsen wang algorithm}
\label{sec-3.2.1}
\subsection{method: determining chemical distance}
\label{sec-3.2.2}
\begin{itemize}

\item review methods in literature\\
\label{sec-3.2.2.1}%
\item proposed trick\\
\label{sec-3.2.2.2}%
\item our method (useful when periodic boundaries)\\
\label{sec-3.2.2.3}%
\item estimated algorithmic complexity\\
\label{sec-3.2.2.4}%
\end{itemize} % ends low level
\subsection{simulation details}
\label{sec-3.2.3}
\begin{itemize}

\item autocorrelation time / independence\\
\label{sec-3.2.3.1}%
\item scaling methods\\
\label{sec-3.2.3.2}%
\end{itemize} % ends low level
\section{Results}
\label{sec-3.3}
\subsection{2D q=1,2,3,4}
\label{sec-3.3.1}
\subsection{3D q=1,2}
\label{sec-3.3.2}
\subsection{4D q=2}
\label{sec-3.3.3}
\chapter{Phase Transitions in Computational Complexity}
\label{sec-4}
\section{Background}
\label{sec-4.1}
\subsection{Constraint Satisfaction Problems (CSP)}
\label{sec-4.1.1}
\begin{itemize}

\item Examples
\label{sec-4.1.1.1}%
\begin{itemize}

\item kSAT\\
\label{sec-4.1.1.1.1}%
\item Graph-coloring\\
\label{sec-4.1.1.1.2}%
\item Spin models\\
\label{sec-4.1.1.1.3}%
\item error-correcting codes\\
\label{sec-4.1.1.1.4}%
\end{itemize} % ends low level

\item Observation of threshold behavior in CSP\\
\label{sec-4.1.1.2}%
\item Difficulties in tackling phase behavior of CSP\\
\label{sec-4.1.1.3}%
\end{itemize} % ends low level
\subsection{Proposal: study complexity of percolation model}
\label{sec-4.1.2}
\section{Percolation}
\label{sec-4.2}
\subsection{The Model}
\label{sec-4.2.1}
\subsection{Background / applications}
\label{sec-4.2.2}
\section{PRAM}
\label{sec-4.3}
\subsection{Applications in comp sci}
\label{sec-4.3.1}
\subsection{PRIORITY CRCW}
\label{sec-4.3.2}
\section{Parallel Algorithm for Percolation}
\label{sec-4.4}
\section{Results}
\label{sec-4.5}
\subsection{D$_2$ vs. p for several system sizes L}
\label{sec-4.5.1}
\subsection{log(D$_2$) vs. log(L)}
\label{sec-4.5.2}
\subsection{Distribution of cluster sizes}
\label{sec-4.5.3}
\begin{itemize}

\item logarithmic or power law? (power law --> algorithm will often fail)\\
\label{sec-4.5.3.1}%
\end{itemize} % ends low level

\end{document}