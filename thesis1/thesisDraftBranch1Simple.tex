% Created 2011-03-16 Wed 16:41
\documentclass[11pt]{article}
\usepackage[utf8]{inputenc}
\usepackage[T1]{fontenc}
\usepackage{fixltx2e}
\usepackage{graphicx}
\usepackage{longtable}
\usepackage{float}
\usepackage{wrapfig}
\usepackage{soul}
\usepackage{textcomp}
\usepackage{marvosym}
\usepackage{wasysym}
\usepackage{latexsym}
\usepackage{amssymb}
\usepackage{hyperref}
\tolerance=1000
\providecommand{\alert}[1]{\textbf{#1}}

\title{thesisDraftBranch1Simple}
\author{Donald Blair}
\date{16 March 2011}

\begin{document}

\maketitle

\setcounter{tocdepth}{3}
\tableofcontents
\vspace*{1cm}
\section{thesis}
\label{sec-1}
\subsection{Melting A}
\label{sec-1_1}
\subsubsection{Background re: melting}
\label{sec-1_1_1}
\begin{itemize}

\item Review of theories of melting, 3D, 2D, bulk\\
\label{sec-1_1_1_1}%
\item Expectations for 2D finite crystallites\\
\label{sec-1_1_1_2}%
\end{itemize} % ends low level
\subsubsection{Experiment of Savage et. al}
\label{sec-1_1_2}
\begin{itemize}

\item Setup\\
\label{sec-1_1_2_1}%
\item Tuneable Depletion potential\\
\label{sec-1_1_2_2}%
\item Results\\
\label{sec-1_1_2_3}%
\end{itemize} % ends low level
\subsubsection{Simulations}
\label{sec-1_1_3}
\begin{itemize}

\item Motivation\\
\label{sec-1_1_3_1}%
\item Justification for using Brownian dynamics\\
\label{sec-1_1_3_2}%
\item GROMACS Simulations\\
\label{sec-1_1_3_3}%
\item Simulated Depletion Potential\\
\label{sec-1_1_3_4}%
\item Simulated Lennard-Jones Potential\\
\label{sec-1_1_3_5}%
\item Results\\
\label{sec-1_1_3_6}%
\item Discussion\\
\label{sec-1_1_3_7}%
\end{itemize} % ends low level
\subsection{Melting B}
\label{sec-1_2}
\subsubsection{Background}
\label{sec-1_2_1}
\begin{itemize}

\item Colloids: macroscopic system analogous to atomic system\\
\label{sec-1_2_1_1}%
\item Experiment by Savage et. al: novel melting kinetics\\
\label{sec-1_2_1_2}%
\item Our hypothesis:  thermally-activated defects enhance melting rate\\
\label{sec-1_2_1_3}%
\item Evidence for hypothesis\\
\label{sec-1_2_1_4}%
\item Background Theory\\
\label{sec-1_2_1_5}%
\end{itemize} % ends low level
\subsubsection{Methods}
\label{sec-1_2_2}
\begin{itemize}

\item Re-analyze data from GROMACS, Part A\\
\label{sec-1_2_2_1}%
\item New Brownian Dynamics Simulation Code\\
\label{sec-1_2_2_2}%
\item Analysis methods\\
\label{sec-1_2_2_3}%
\end{itemize} % ends low level
\subsubsection{Results / Figures}
\label{sec-1_2_3}
\begin{itemize}

\item N vs t\\
\label{sec-1_2_3_1}%
\item Order vs. N\\
\label{sec-1_2_3_2}%
\item Breakdown by layers\\
\label{sec-1_2_3_3}%
\item Average disclination charge\\
\label{sec-1_2_3_4}%
\item Phase diagram for various ranges of potential\\
\label{sec-1_2_3_5}%
\end{itemize} % ends low level
\subsubsection{Discussion}
\label{sec-1_2_4}
\subsection{Diameter of Random Clusters}
\label{sec-1_3}
\subsubsection{Introduction}
\label{sec-1_3_1}
\subsubsection{Potts Model \cite{Wu82}}
\label{sec-1_3_2}
\subsubsection{Chemical Distance}
\label{sec-1_3_3}
\subsubsection{Diameter}
\label{sec-1_3_4}
\subsubsection{Swendesen-Wang Algorithm}
\label{sec-1_3_5}
\subsubsection{Determining the Chem Distance and Diameter}
\label{sec-1_3_6}
\subsubsection{Procedure for $q>1$}
\label{sec-1_3_7}
\subsubsection{Procedure for $q>1$}
\label{sec-1_3_8}
\subsubsection{Simulation Details}
\label{sec-1_3_9}
\subsection{Phase Transitions in Computational Complexity}
\label{sec-1_4}
\subsubsection{Background}
\label{sec-1_4_1}
\begin{itemize}

\item Constraint Satisfaction Problems (CSP)\\
\label{sec-1_4_1_1}%
\item Proposal: study complexity of percolation model\\
\label{sec-1_4_1_2}%
\end{itemize} % ends low level
\subsubsection{Percolation}
\label{sec-1_4_2}
\begin{itemize}

\item The Model\\
\label{sec-1_4_2_1}%
\item Background / applications\\
\label{sec-1_4_2_2}%
\end{itemize} % ends low level
\subsubsection{PRAM}
\label{sec-1_4_3}
\begin{itemize}

\item Applications in comp sci\\
\label{sec-1_4_3_1}%
\item PRIORITY CRCW\\
\label{sec-1_4_3_2}%
\end{itemize} % ends low level
\subsubsection{Parallel Algorithm for Percolation}
\label{sec-1_4_4}
\subsubsection{Results}
\label{sec-1_4_5}
\begin{itemize}

\item D$_2$ vs. p for several system sizes L\\
\label{sec-1_4_5_1}%
\item log(D$_2$) vs. log(L)\\
\label{sec-1_4_5_2}%
\item Distribution of cluster sizes\\
\label{sec-1_4_5_3}%
\end{itemize} % ends low level
\subsubsection{Discussion}
\label{sec-1_4_6}

\end{document}
