% Created 2010-07-04 Sun 14:53
\documentclass{umthesis}


\title{Draft of diameter paper}
\author{Don Blair}
\date{04 July 2010}

\begin{document}



\setcounter{tocdepth}{5}
\tableofcontents
\vspace*{1cm}
\title{ \emph{Physicae Auscultationes}}
\author{D. W. Blair}
\date{September 2010}

\copyrightyear{2010}
\bachelors{B.Sc.}{University of Massachusetts Amherst}
\masters{M.Sc.}{University of Massachusettds Amherst}
%\secondmasters{M.Ed.}{Antioch College}
%\priordoctorate{M.D.}{University of Never-never-land}
% \committeechair{B. B. Bahh}
\cochairs{B. B. Bahh}{I. M. A. Wolf}
\firstreader{Little Bo Peep}
\secondreader{R. U. Sheepish}
\thirdreader{Bill Shepherd}
\fourthreader{Mary Lamb}   % Optional
%\fifthreader{}            % Optional
%\sixthreader{}            % Optional
\departmentchair{Don Candela}
\departmentname{Physics Department}


%\frontmatter
%\maketitle
%\copyrightpage
%\signaturepage


%\begin{dedication}
%  \begin{center}
%    \emph{To those little lost sheep.}
%  \end{center}
%\end{dedication}

%\chapter{Acknowledgments}
%Acknowledgements.

%\begin{abstract} 
%Test of abstract
%\end{abstract}


%\tableofcontents                % Table of contents -- ADD BACK (and uncomment 'OPTIONS' line at top) for umthesis style TOC
%\settocdepth{subparagraph}

%\listoftables                   % List of Tables -- ADD BACK IN
%\listoffigures                  % List of Figures -- ADD BACK IN
\mainmatter

%\unnumberedchapter{Introduction}
%Why on earth do I want to study sheep anyway?


\chapter{Melting A}
\label{sec-1}
\section{Background re: melting}
\label{sec-1.1}
\subsection{Theories of melting, 3D, 2D, bulk}
\label{sec-1.1.1}
\subsubsection{3D crystallites w/ stable surfaces melt from within via Born melting}
\label{sec-1.1.1.1}

In this case, melting can be viewed as nucleation and growth of fluid phase within the solid.
\paragraph{or yet another structure.}
\label{sec-1.1.1.1.1}
\begin{itemize}

\item or even another\\
\label{sec-1.1.1.1.1.1}%
\end{itemize} % ends low level
\subsubsection{2D large crystallites melt by two-step process via hexatic phase}
\label{sec-1.1.1.2}
\subsubsection{2D finite crystallites melt from perimeter}
\label{sec-1.1.1.3}
\paragraph{if melt from perimeter, dN/dt goes as $N^{1/2}$}
\label{sec-1.1.1.3.1}
\subsection{Expectations for 2D finite crystallites}
\label{sec-1.1.2}
\section{Experiment of Savage et. al}
\label{sec-1.2}
\subsection{Setup}
\label{sec-1.2.1}
\subsection{Tunable Depletion potential}
\label{sec-1.2.2}
\subsection{Results}
\label{sec-1.2.3}
\subsubsection{N vs. t}
\label{sec-1.2.3.1}
\subsubsection{$< psi6 >^2$ vs. N}
\label{sec-1.2.3.2}
\subsubsection{$C_6$ vs. N, by layer}
\label{sec-1.2.3.3}
\subsubsection{No dependence of fast-melting feature on initial cluster size or melting rate}
\label{sec-1.2.3.4}
\section{Simulations}
\label{sec-1.3}
\subsection{Motivation}
\label{sec-1.3.1}
\subsection{GROMACS System}
\label{sec-1.3.2}
\subsection{Brownian dynamics}
\label{sec-1.3.3}
\subsection{Simulated Depletion Potential}
\label{sec-1.3.4}
\subsection{Simulated Lennard-Jones Potential}
\label{sec-1.3.5}
\subsection{Results}
\label{sec-1.3.6}
\subsubsection{N vs. t}
\label{sec-1.3.6.1}
\subsubsection{$< psi6 >^2$ vs. N}
\label{sec-1.3.6.2}
\subsubsection{$C_6$ vs. N, by layer}
\label{sec-1.3.6.3}
\subsubsection{mean-square fluctuations in bond lengths}
\label{sec-1.3.6.4}
\subsubsection{N vs. t for Lennard-Jones potential}
\label{sec-1.3.6.5}
\subsubsection{Phase diagram showing lack of fluid phase with short-range potential}
\label{sec-1.3.6.6}
\subsection{Discussion}
\label{sec-1.3.7}
\chapter{Melting B}
\label{sec-2}
\section{Background}
\label{sec-2.1}
\subsection{Hypothesis: thermally-activated defects enhance melting rate in short-range 2D system}
\label{sec-2.1.1}
\section{Simulation Methods}
\label{sec-2.2}
\subsection{Gromacs system}
\label{sec-2.2.1}

Here's a good test. \cite{Deng2009}
\subsection{Brownian Dynamics}
\label{sec-2.2.2}
\subsection{Characteristics of Simulated Depletion Potential}
\label{sec-2.2.3}
\subsection{Initial configurations}
\label{sec-2.2.4}
\section{Results}
\label{sec-2.3}
\subsection{N vs t}
\label{sec-2.3.1}
\subsection{Order vs. N}
\label{sec-2.3.2}
\subsection{Breakdown by layers}
\label{sec-2.3.3}
\section{Conclusions}
\label{sec-2.4}
\chapter{Diameter of Random Clusters}
\label{sec-3}
\section{Background}
\label{sec-3.1}
\section{Simulations}
\label{sec-3.2}
\section{Results}
\label{sec-3.3}
\chapter{Phase Transitions in Computational Complexity}
\label{sec-4}
\section{Background}
\label{sec-4.1}
\subsection{Constraint Satisfaction Problems (CSP)}
\label{sec-4.1.1}
\subsubsection{Examples}
\label{sec-4.1.1.1}
\paragraph{kSAT}
\label{sec-4.1.1.1.1}
\paragraph{Graph-coloring}
\label{sec-4.1.1.1.2}
\paragraph{Spin models}
\label{sec-4.1.1.1.3}
\paragraph{error-correcting codes}
\label{sec-4.1.1.1.4}
\subsubsection{Observation of threshold behavior in CSP}
\label{sec-4.1.1.2}
\subsubsection{Difficulties in tackling phase behavior of CSP}
\label{sec-4.1.1.3}
\subsection{Proposal: study complexity of percolation model}
\label{sec-4.1.2}
\section{Percolation}
\label{sec-4.2}
\subsection{The Model}
\label{sec-4.2.1}
\subsection{Background / applications}
\label{sec-4.2.2}
\section{PRAM}
\label{sec-4.3}
\subsection{Applications in comp sci}
\label{sec-4.3.1}
\subsection{PRIORITY CRCW}
\label{sec-4.3.2}
\section{Parallel Algorithm for Percolation}
\label{sec-4.4}
\section{Results}
\label{sec-4.5}
\subsection{D$_2$ vs. p for several system sizes L}
\label{sec-4.5.1}
\subsection{log(D$_2$) vs. log(L)}
\label{sec-4.5.2}
\subsection{Distribution of cluster sizes}
\label{sec-4.5.3}
\subsubsection{logarithmic or power law? (power law --> algorithm will often fail)}
\label{sec-4.5.3.1}



\bibliographystyle{plain}
\bibliography{/home/dwblair/Dropbox/dwbdocs/physics/writing/bibfiles/combo}

 

\end{document}