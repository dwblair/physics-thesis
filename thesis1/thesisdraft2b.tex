% Created 2010-07-01 Thu 15:10
\documentclass{umthesis}


\title{thesisdraft2b}
\author{Don Blair}
\date{01 July 2010}

\begin{document}



\setcounter{tocdepth}{3}
\tableofcontents
\vspace*{1cm}
\chapter{Melting A}
\label{sec-1}
\section{Background re: melting}
\label{sec-1.1}
\subsection{Theories of melting, 3D, 2D, bulk}
\label{sec-1.1.1}
\begin{itemize}

\item 3D crystallites w/ stable surfaces melt from within via Born melting\\
\label{sec-1.1.1.1}%
In this case, melting can be viewed as nucleation and growth of fluid phase within the solid.

\item 2D large crystallites melt by two-step process via hexatic phase\\
\label{sec-1.1.1.2}%
\item 2D finite crystallites melt from perimeter
\label{sec-1.1.1.3}%
\begin{itemize}

\item if melt from perimeter, dN/dt goes as $N^{1/2}$\\
\label{sec-1.1.1.3.1}%
\end{itemize} % ends low level
\end{itemize} % ends low level
\subsection{Expectations for 2D finite crystallites}
\label{sec-1.1.2}
\section{Experiment of Savage et. al}
\label{sec-1.2}
\subsection{Setup}
\label{sec-1.2.1}
\subsection{Tunable Depletion potential}
\label{sec-1.2.2}
\subsection{Results}
\label{sec-1.2.3}
\begin{itemize}

\item N vs. t\\
\label{sec-1.2.3.1}%
\item $< psi6 >^2$ vs. N\\
\label{sec-1.2.3.2}%
\item $C_6$ vs. N, by layer\\
\label{sec-1.2.3.3}%
\item No dependence of fast-melting feature on initial cluster size or melting rate\\
\label{sec-1.2.3.4}%
\end{itemize} % ends low level
\section{Simulations}
\label{sec-1.3}
\subsection{Motivation}
\label{sec-1.3.1}
\subsection{GROMACS System}
\label{sec-1.3.2}
\subsection{Brownian dynamics}
\label{sec-1.3.3}
\subsection{Simulated Depletion Potential}
\label{sec-1.3.4}
\subsection{Simulated Lennard-Jones Potential}
\label{sec-1.3.5}
\subsection{Results}
\label{sec-1.3.6}
\begin{itemize}

\item N vs. t\\
\label{sec-1.3.6.1}%
\item $< psi6 >^2$ vs. N\\
\label{sec-1.3.6.2}%
\item $C_6$ vs. N, by layer\\
\label{sec-1.3.6.3}%
\item mean-square fluctuations in bond lengths\\
\label{sec-1.3.6.4}%
\item N vs. t for Lennard-Jones potential\\
\label{sec-1.3.6.5}%
\item Phase diagram showing lack of fluid phase with short-range potential\\
\label{sec-1.3.6.6}%
\end{itemize} % ends low level
\subsection{Discussion}
\label{sec-1.3.7}
\chapter{Melting B}
\label{sec-2}
\section{Background}
\label{sec-2.1}
\subsection{Hypothesis: thermally-activated defects enhance melting rate in short-range 2D system}
\label{sec-2.1.1}
\section{Simulation Methods}
\label{sec-2.2}
\subsection{Gromacs system}
\label{sec-2.2.1}
\subsection{Brownian Dynamics}
\label{sec-2.2.2}
\subsection{Characteristics of Simulated Depletion Potential}
\label{sec-2.2.3}
\subsection{Initial configurations}
\label{sec-2.2.4}
\section{Results}
\label{sec-2.3}
\subsection{N vs t}
\label{sec-2.3.1}
\subsection{Order vs. N}
\label{sec-2.3.2}
\subsection{Breakdown by layers}
\label{sec-2.3.3}
\section{Conclusions}
\label{sec-2.4}
\chapter{Diameter of Random Clusters}
\label{sec-3}
\section{Background}
\label{sec-3.1}
\section{Simulations}
\label{sec-3.2}
\section{Results}
\label{sec-3.3}
\chapter{Phase Transitions in Computational Complexity}
\label{sec-4}
\section{Background}
\label{sec-4.1}
\subsection{Constraint Satisfaction Problems (CSP)}
\label{sec-4.1.1}
\begin{itemize}

\item Examples
\label{sec-4.1.1.1}%
\begin{itemize}

\item kSAT\\
\label{sec-4.1.1.1.1}%
\item Graph-coloring\\
\label{sec-4.1.1.1.2}%
\item Spin models\\
\label{sec-4.1.1.1.3}%
\item error-correcting codes\\
\label{sec-4.1.1.1.4}%
\end{itemize} % ends low level

\item Observation of threshold behavior in CSP\\
\label{sec-4.1.1.2}%
\item Difficulties in tackling phase behavior of CSP\\
\label{sec-4.1.1.3}%
\end{itemize} % ends low level
\subsection{Proposal: study complexity of percolation model}
\label{sec-4.1.2}
\section{Percolation}
\label{sec-4.2}
\subsection{The Model}
\label{sec-4.2.1}
\subsection{Background / applications}
\label{sec-4.2.2}
\section{PRAM}
\label{sec-4.3}
\subsection{Applications in comp sci}
\label{sec-4.3.1}
\subsection{PRIORITY CRCW}
\label{sec-4.3.2}
\section{Parallel Algorithm for Percolation}
\label{sec-4.4}
\section{Results}
\label{sec-4.5}
\subsection{D$_2$ vs. p for several system sizes L}
\label{sec-4.5.1}
\subsection{log(D$_2$) vs. log(L)}
\label{sec-4.5.2}
\subsection{Distribution of cluster sizes}
\label{sec-4.5.3}
\begin{itemize}

\item logarithmic or power law? (power law --> algorithm will often fail)\\
\label{sec-4.5.3.1}%
\bibliographystyle{plain}
\bibliography{/home/dwblair/Dropbox/dwbdocs/physics/writing/bibfiles/combo}

 
\end{itemize} % ends low level

\end{document}