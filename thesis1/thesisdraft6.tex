% Created 2010-07-06 Tue 12:20
\documentclass{umthesis} 
\usepackage{tocloft}
%\renewcommand{\contentsname}{Thesis Outline (\today) }
% remove following commands when publish thesis:
\chead{Thesis Outline, Don Blair (\today)}
\setlength{\cftsubsubsecnumwidth}{1.8em} 
\setlength{\cftparaindent}{9.3em}
\setlength{\cftparanumwidth}{1.5em} 
\renewcommand{\thesubsubsection}{\alph{subsubsection}.}
\renewcommand{\theparagraph}{$\bullet$ }
%\renewcommand{\cftdotsep}{10000}
% remove above commands when publishing thesis




\title{Draft of diameter paper}
\author{Don Blair}
\date{06 July 2010}

\begin{document}



\setcounter{tocdepth}{5}
\tableofcontents
\vspace*{1cm}
\title{ \emph{Physicae Auscultationes}}
\author{D. W. Blair}
\date{September 2010}

\copyrightyear{2010}
\bachelors{B.Sc.}{University of Massachusetts Amherst}
\masters{M.Sc.}{University of Massachusettds Amherst}
%\secondmasters{M.Ed.}{Antioch College}
%\priordoctorate{M.D.}{University of Never-never-land}
% \committeechair{B. B. Bahh}
\cochairs{B. B. Bahh}{I. M. A. Wolf}
\firstreader{Little Bo Peep}
\secondreader{R. U. Sheepish}
\thirdreader{Bill Shepherd}
\fourthreader{Mary Lamb}   % Optional
%\fifthreader{}            % Optional
%\sixthreader{}            % Optional
\departmentchair{Don Candela}
\departmentname{Physics Department}


%\frontmatter
%\maketitle
%\copyrightpage
%\signaturepage


%\begin{dedication}
%  \begin{center}
%    \emph{To those little lost sheep.}
%  \end{center}
%\end{dedication}

%\chapter{Acknowledgments}
%Acknowledgements.

%\begin{abstract} 
%Test of abstract
%\end{abstract}


%\tableofcontents                % Table of contents -- ADD BACK (and uncomment 'OPTIONS' line at top) for umthesis style TOC
%\settocdepth{subparagraph}

%\listoftables                   % List of Tables -- ADD BACK IN
%\listoffigures                  % List of Figures -- ADD BACK IN
\mainmatter

%\unnumberedchapter{Introduction}
%Why on earth do I want to study sheep anyway?


\chapter{Melting A}
\label{sec-1}
\section{Background re: melting}
\label{sec-1.1}
\subsection{Theories of melting, 3D, 2D, bulk}
\label{sec-1.1.1}
\subsubsection{3D crystallites w/ stable surfaces melt from within via Born melting}
\label{sec-1.1.1.1}

     \texttt{CLOSED:} \textit{2010-07-04 Sun 15:28}\newline
In this case, melting can be viewed as nucleation and growth of fluid phase within the solid.
\paragraph{or yet another structure.}
\label{sec-1.1.1.1.1}
\begin{itemize}

\item or even another\\
\label{sec-1.1.1.1.1.1}%
\end{itemize} % ends low level
\subsubsection{2D large crystallites melt by two-step process via hexatic phase}
\label{sec-1.1.1.2}
\subsubsection{2D finite crystallites melt from perimeter}
\label{sec-1.1.1.3}
\paragraph{if melt from perimeter, dN/dt goes as $N^{1/2}$}
\label{sec-1.1.1.3.1}
\subsection{Expectations for 2D finite crystallites}
\label{sec-1.1.2}
\section{Experiment of Savage et. al}
\label{sec-1.2}
\subsection{Setup}
\label{sec-1.2.1}
\subsection{Tuneable Depletion potential}
\label{sec-1.2.2}
\subsection{Results}
\label{sec-1.2.3}
\subsubsection{N vs. t}
\label{sec-1.2.3.1}
\subsubsection{$< psi6 >^2$ vs. N}
\label{sec-1.2.3.2}
\subsubsection{$C_6$ vs. N, by layer}
\label{sec-1.2.3.3}
\subsubsection{No dependence of fast-melting feature on initial cluster size or melting rate}
\label{sec-1.2.3.4}
\section{Simulations}
\label{sec-1.3}
\subsection{Motivation}
\label{sec-1.3.1}
\subsubsection{Rule out any hydrodynamic effects causing fast-melting}
\label{sec-1.3.1.1}
\subsubsection{Determine whether range of potential plays role in fast melting}
\label{sec-1.3.1.2}
\subsubsection{Justification for using Brownian dynamics}
\label{sec-1.3.1.3}
\subsection{GROMACS Simulations}
\label{sec-1.3.2}
\subsubsection{Brownian dynamics option}
\label{sec-1.3.2.1}
\subsubsection{Interparticle 'depletion' potential}
\label{sec-1.3.2.2}
\paragraph{Mimics that in experiment}
\label{sec-1.3.2.2.1}
\paragraph{$U(r)=0$ for $r > r_0$}
\label{sec-1.3.2.2.2}
\paragraph{$U(r)=4/(10r-9)^{12} -  400 a_0 (r-r_0)^2$ for $r \le r_0$}
\label{sec-1.3.2.2.3}

with the first term resembling hard sphere repulsion and the second term  representing a two-body depletion potential. The parameters $a_0=1.0$ and $r_0=1.1$ were chosen to allow for  a potential with a narrow width compared to the particle diameter. This potential has an effective particle diameter $\sigma=1.0$,  a width equal to $0.1$ and an equilibrium inter-particle spacing $a \approx 1.01637$
\subsubsection{Temperature}
\label{sec-1.3.2.3}
\subsubsection{Effective well depth: $3.5 k_B T$}
\label{sec-1.3.2.4}
\subsubsection{Time step: $2.5 \times 10^{-5}$ (in GROMACS time units)}
\label{sec-1.3.2.5}
\subsubsection{$N=100$ particles}
\label{sec-1.3.2.6}
\subsubsection{periodic box of size $L = 18.0 \sigma$}
\label{sec-1.3.2.7}
\subsubsection{particle area fraction of $24\%$}
\label{sec-1.3.2.8}

\end{document}