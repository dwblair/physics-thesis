% Created 2010-07-01 Thu 14:23
\documentclass{article}


\title{thesisdraft3}
\author{Don Blair}
\date{01 July 2010}

\begin{document}



\setcounter{tocdepth}{5}
\tableofcontents

\section{test 1}

\subsection{test 2}

\subsubsection{test 3}

\section{Melting A}

\subsection{Background re: melting}

\subsubsection{Theories of melting, 3D, 2D, bulk}




\subsubsection{ 3D crystallites w/ stable surfaces melt from within via Born melting}
\subsubsection{ 2D large crystallites melt by two-step process via hexatic phase}

\subsubsection{ 2D finite crystallites melt from perimeter



\subsubsection{ if melt from perimeter, dN/dt goes as $N^{1/2}$}


\subsubsection{Expectations for 2D finite crystallites}

\subsection{Experiment of Savage et. al}

\subsubsection{Setup}

\subsubsection{Tunable Depletion potential}

\subsubsection{Results}



\subsubsection{ N vs. t}

\subsubsection{ $< psi6 >^2$ vs. N}

\subsubsection{ $C_6$ vs. N, by layer}

\subsubsection{ No dependence of fast-melting feature on initial cluster size or melting rate}


\subsection{Simulations}

\subsubsection{Motivation}

\subsubsection{GROMACS System}

\subsubsection{Brownian dynamics}

\subsubsection{Simulated Depletion Potential}

\subsubsection{Simulated Lennard-Jones Potential}

\subsubsection{Results}



\subsubsection{ N vs. t}

\subsubsection{ $< psi6 >^2$ vs. N}

\subsubsection{ $C_6$ vs. N, by layer}

\subsubsection{ mean-square fluctuations in bond lengths}

\subsubsection{ N vs. t for Lennard-Jones potential}

\subsubsection{ Phase diagram showing lack of fluid phase with short-range potential}


\subsubsection{Discussion}

\section{Melting B}

\subsection{Background}

\subsubsection{Hypothesis: thermally-activated defects enhance melting rate in short-range 2D system}

\subsection{Simulation Methods}

\subsubsection{Gromacs system}

\subsubsection{Brownian Dynamics}

\subsubsection{Characteristics of Simulated Depletion Potential}

\subsubsection{Initial configurations}

\subsection{Results}

\subsubsection{N vs t}

\subsubsection{Order vs. N}

\subsubsection{Breakdown by layers}

\subsection{Conclusions}

\section{Diameter of Random Clusters}

\subsection{Background}

\subsection{Simulations}

\subsection{Results}

\section{Phase Transitions in Computational Complexity}

\subsection{Background}

\subsubsection{Constraint Satisfaction Problems (CSP)}



\subsubsection{ Examples



\subsubsection{ kSAT}

\subsubsection{ Graph-coloring}

\subsubsection{ Spin models}

\subsubsection{ error-correcting codes}



\subsubsection{ Observation of threshold behavior in CSP}

\subsubsection{ Difficulties in tackling phase behavior of CSP}

 % ends low level
\subsubsection{Proposal: study complexity of percolation model}

\subsection{Percolation}

\subsubsection{The Model}

\subsubsection{Background / applications}

\subsection{PRAM}

\subsubsection{Applications in comp sci}

\subsubsection{PRIORITY CRCW}

\subsection{Parallel Algorithm for Percolation}

\subsection{Results}

\subsubsection{D$_2$ vs. p for several system sizes L}

\subsubsection{log(D$_2$) vs. log(L)}

\subsubsection{Distribution of cluster sizes}



\subsubsection{ logarithmic or power law? (power law --> algorithm will often fail)}

\bibliographystyle{plain}
\bibliography{/home/dwblair/Dropbox/dwbdocs/physics/writing/bibfiles/combo}

 
 % ends low level

\end{document}