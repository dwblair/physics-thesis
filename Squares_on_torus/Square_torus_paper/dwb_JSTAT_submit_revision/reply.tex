\title{
Reply to Editor's Report \\ 
JSTAT 003P 1111 \\
Title: Packing Squares in a Torus}
\author{
       Donald W. Blair \\
       Christian D. Santangelo \\
Jon Machta
}
\date{\today}

\documentclass[12pt]{article}

\begin{document}
\setlength{\parindent}{0.0in}
\setlength{\parskip}{4mm}
%\maketitle

%\begin{abstract}
%This is the paper's abstract \ldots
%\end{abstract}
{\bf Reply to Editor's Report on JSTAT 003P 1111}

{\bf Authors}: Donald W. Blair, Christian D. Santangelo, Jon Machta \\
{\bf Title}: Packing Squares in a Torus

\setlength{\parskip}{4mm}
The authors have read the Editor's Report, and we appreciate the thoughtful comments made therein.  Below, we outline our attempts to address each of the referee's concerns.  Each item corresponds to the numbered notes in the editor's report.

\begin{quote} 
1) I suggest to improve the motivation of the work in the Introduction.
\end{quote}
In order to better situate our results within physics, we have added a paragraph to the introduction (the fourth paragraph, which begins ``What is the effect of an external potential on hard objects?'').  We have also included an additional reference to the computational physics literature: reference 20, Wojciechowski and Frenkel (2004). 


\begin{quote} 
2) In the first line of II/A/2 the authors write about the case $N=n_2^2+n_4^2$, however, in the previous subsection they have proved that this is the case of perfect packing. I think this is a mistake, they should write $N=n_2^2+n_4^2-k$.
\end{quote}

We agree, and have added the ``-k'' to the relevant equation.

\begin{quote}
3) Moreover, I feel that the title of the subsection II/A/2 is ambiguous, at the first sight the reader does not understand how can be a packing is density-one if it contains vacancies. I suggest the simple title ``vacancies'' or ``vacancies in density-one packings'' or somethig similar.
\end{quote}

We have changed this subsection's title to ``Lattice packings with vacancies'' in order to address this ambiguity.

\begin{quote}
4) In the first line of II/A/3 the authors write $d<1$. It is a mistake, correctly it would be $d>1$.
\end{quote}

We agree, and we have changed the relevant sentence to read: ``Next we consider Bravais lattice solutions that have density less than one -- that is, packings with gap $d-1 > 0$.''

\begin{quote}
5) The reference [22] is incomplete, this famous article has two authors, R.
L. Graham beside of Paul Erd\H{o}s.
\end{quote}

We have added R.L. Graham as second author for this reference.

\begin{quote}
Q1) In II/A/2 the authors speak about density-one packings with a vacancy. It has no figure about this packing but the text suggests that the unit squares are situated precisely at the lattice points of a Bravais lattice except of one lattice point at which the vacancy is situated. Why do not the nearest neighbours of the vacancie move partially into the empty place? I feel that the empty space covered by the vacancie should be distributed randomly along a line (or column), breaking the translational invariance of the Bravais lattice. Am I right or not? I suggest to clarify this issue in the text.
\end{quote}

The referee is correct: the squares in a row with a vacancy are indeed able to undergo continuous translations within the row. In order to clarify this point, we have included an additional figure (Figure 4), and we have revised the relevant text to read as follows: 

``The simplest way to produce candidates for a densest packing for $N=n_2^2 + n_4^2-k$ is to remove $k$ squares from a density-one packing; indeed, our numerical results suggest that for several values of $N$, the densest packing is a density-one packing with one missing square.  This is indicated in the Comment column in the Table using the notation $n_1^2-1$ or $n_1^2+ n_2^2-1$, depending on whether they are generated by removing $1$ square from $N$ a square integer, or a sum of two square integers respectively.  As is demonstrated in Fig. 4 for the case of $N=2^2+2^2-1=7$, such vacancies allow for continuous displacement of other squares within a row, leading to a finite entropy for such configurations.  Other examples include $N=3$ and $15$.''

Thanks again for all of the helpful feedback.

Best Regards,\\
Donald Blair\\
Christian Santangelo\\
Jon Machta\\
\end{document}
