\section{Introduction}

Understanding the dense packings of hard particles has yielded essential insights into the structure of materials \cite{Bernal2007} \cite{Zallen1983} \cite{Torquato2002} \cite{Chaikin2000}, granular media \cite{Torquato2002} \cite{Mehta1994}, number theory \cite{Cohn2008} \cite{Conway1999}, biology 
%\cite{Liang2001}
\cite{Gevertz2008} \cite{Purohit2003}, and computer science \cite{Johnson1974} \cite{Lodi2002}. This understanding has been hard won: millennia can elapse between a conjecture and its proof. This is best exemplified by sphere packing, for which a proof had not been found until 1998 \cite{Hales2011}.

%Note: for references in this above paragraph, see paper by Torguato and Jiao

% Note: I changed this from 1999 to 1998, given http://en.wikipedia.org/wiki/Thomas_Callister_Hales and other sources
Recent experimental advances have allowed the development of (nearly) hard colloids that, for entropic reasons, manifest the densest sphere packing [REF??]. The attempt to obtain a deep understanding of liquid crystal mesophases has suggested study of the dense packing of anisotropic particles. This has led to recent explorations of the packing of ellipsoids \cite{Donev2004}, polyhedra \cite{ROAN} \cite{BAKER}, and polygons \cite{Jansson2006} \cite{STROOBANTS}.

One of the simplest regular polygons one can pack in two dimensions is the square. On the plane, the densest packing is trivial -- a square lattice of squares. Monte Carlo simulations of squares at finite pressure, however, have also found a tetratic phase \cite{Donev2006b}, and experiments with hard colloidal squares have found, rather than the tetratic phase, a rhombic crystal having a different symmetry than the square \cite{Zhao2011}. Even the dense packing of a finite number of squares can be more complicated than naive considerations would indicate -- in fact, this problem has been shown to be NP-hard \cite{Leung1990}. The densest known packings of squares inside a larger square can be quite complex \cite{ERDOS1975} \cite{Friedman2002} when the number of squares is not a perfect square. Higher packing densities than that of a simple, square lattice with defects can be achieved through configurations in which some of the squares are rotated and shifted with respect to the simple square lattice \cite{Friedman2002}.

In this paper, we study an even simpler packing problem: finding the densest packings of squares in a torus -- that is, a larger square with periodic boundary conditions. Even with the additional translation symmetry afforded by packing squares in a torus rather than in a square, the resulting dense packings in the torus can still be far from simple. As in many other mathematical packing problems, the strategy here is to search for the smallest area torus that can accommodate a fixed number of squares $N$.  We use a combination of analytic and Monte Carlo simulated annealing techniques to accomplish this, and our results can be summarized as follows: we find that whenever $N$ can be expressed as the sum of two integers -- $N=n_1^2+n_2^2$ -- the densest possible packing is a density one packing with squares arranged in rows that are at an orientation of tan$^{-1}(n_2/n_1)$ relative to the underlying torus.  For other cases of $N$ up to $N=27$, we find a surprisingly rich collection of dense packing structures. For $N$ = 6,11,14, and 27, we believe that the optimal packing is also a commensurate Bravais Lattice packing with density $N/(N+1)$ and resembles a bricklayer pattern with periodic gaps.  For $N=$ 12,21,22 and 23, we find that the densest configurations are non-Bravais lattice packings, including both regular lattices of holes and of skewed squares. These results are summarized in Table I within Section II.

In Section II, we present a summary analysis of the various structures we found for $N$ up to 27, including both commensurate Bravais lattice solutions and non-Bravais lattice solutions.  The packing motifs we found through analytic and numerical means are illustrated in this section via both drawn figures and images generated by our numerical simulations. In Section III, we provide details of our numerical experiments for the hard square system. Section IV is a discussion of our results, including an analysis of the entropy of density one packings and the rotational invariance of such packings as $N \rightarrow \infty$. [DOES THIS LAST SENTENCE WORK?]

