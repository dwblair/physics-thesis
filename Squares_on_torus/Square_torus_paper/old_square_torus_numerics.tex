\section{Analysis of Packings}
\label{sec:analytics}

In this section we give an analytic treatment of square packings on a torus.  We first describe a set of solutions in which the squares lie on a Bravais lattice and then turn to more complicated cases. The density one solutions are optimal by construction and all of the other solutions are conjectured to be optimal. The numerical results of Sec. \ref{sec:numerical} guided us to the conjectured solutions and the fact that long annealing runs consistently produced these solutions gives us some confidence that they are optimal.  Our conjectures for the first 27 values of the number of squares are summarized in Table \ref{table}.

\subsection{Commensurate Bravais Lattice Solutions}
In this section we consider a class of Bravais lattice configurations that include all of the density one packings and other solutions we have found for  $N \leq 27$.   In all of these packings the squares are lined up in rows.  For purposes of the analysis, we assume these rows are aligned along the x-axis. Thus one of the primitive vectors of the lattice of squares is $\ax=\hat{{\bf x}}$.   The second primitive vector is taken to have the form $\ay=c \hat{{\bf x}} + |d| \hat{{\bf y}}$ with $-1< c <1$ and  $d\geq 1$.  The primitive vectors of the torus, $\Ax$ and $\Ay$, need not be aligned with the primitive vectors of the squares.     The requirement that the squares pack periodically on the torus is equivalent to saying that the lattice of squares is commensurate with the larger square lattice of the torus.  That is, there exist  integers $n_1$, $n_2$, $n_3$ and $n_4$ such that the torus primitive vectors $\Ax$ and $\Ay$ are given by
\begin{eqnarray} 
\label{eqn:Ana}
\Ax&=& n_1 \ax + n_2 \ay \nonumber \\ 
\Ay&=&n_3 \ax + n_4 \ay.
\end{eqnarray}
In addition, we require that the torus primitive vectors are of equal length,
\begin{equation}
\label{eqn:normal}
|\Ax|=|\Ay| ,
\end{equation}
and orthogonal,
\begin{equation}
\label{eqn:ortho}
\Ax \cdot \Ay = 0 .
\end{equation}
These conditions are uniquely solved by
\begin{eqnarray}\label{eq:gap}
c &=& - \frac{n_1 n_2 + n_3 n_4}{n_2^2 + n_4^2}\\
d &=& \frac{| n_1 n_4 - n_2 n_3|}{n_2^2 + n_4^2} \nonumber
\end{eqnarray}
The number of squares $N$ packed on the torus is the number of lattice points of the squares lattice in a unit cell of the torus lattice
\begin{equation}
\label{eqn:N}
N = |n_1 n_4 - n_2 n_3|,
\end{equation}
and the areal density of the squares $\rho$ is given by
\begin{equation}
\label{ }
\rho=N/|\Ax \times \Ay| = 1/d.
\end{equation}

\subsubsection{Density one (and related) packings}
There are two classes of density one solutions.  The first is the perfect square packing.  Here $c=0$, $d=1$, $n_1=n_4=\sqrt{N}$ and $n_2=n_3=0$.  This simple packing is shown in Fig \ref{fig:N9} for the case of $N=9$.  The perfect square packing is obviously optimal for all perfect squares.  Note that on the torus, each of the $n_1$ rows (or columns, but not both) may be arbitrarily displaced relative to the other rows (columns) without disturbing density of the packing or its periodicity.  Thus the perfect square packings have finite entropy.

There is another class of density one packings, however, when the lattice of squares is tilted with respect to the primitive vectors of the torus. Setting $d=1$ and $c=0$ in Eq. (\ref{eq:gap}), we find $N=n_2^2 + n_4^2$, $n_1=-n_4$ and $n_2 = n_3$.  Clearly, these density one, tilted square lattice solutions are optimal for all $N$ that are sums of two square integers.  Fig \ref{fig:bravais} shows the case $N=10$ ($n_1=3$ and $n_2=1$).

The converse of this is also true: $N$ is a sum of two squares whenever we have a density one packing of squares in the torus.
To prove this, first note that every square in a $\rho=1$ packing must have at least four other squares bordering it along a finite segment length, forcing all $N$ squares to share the same orientation.  Now consider three squares in mutual contact with each other, a configuration that must exist if the packing has no gaps. Two of those squares must be aligned in a row, as shown in Fig. X. In order to eliminate gaps in the packing, those three squares define a set of rows that the entire packing must respect. The periodicity of toroidal boundary conditions on a set of rows now ensures that $N$ can be expressed as $n_2^2 + n_4^2$ via the Pythagorean theorem (see Fig. \ref{fig:pythagorus}).

Table \ref{table} indicates the values of $N$ that are either perfect squares or sums of two squares.  In the Comment column of the Table these values are indicated either by the forms $n_1^2$ or $n_1^2+ n_2^2$, respectively. Though it appears that density one packings are relatively common from the table, in fact we know that the frequency of numbers that are the sums of two square integers scales as $1/\sqrt{\ln N}$ for large $N$ \cite{Berndt1993} \cite{Landau1909}.
% for more references on density of sums of squares, see http://mathworld.wolfram.com/SumofSquaresFunction.html
 As a consequence of this, the frequency of density one packings also vanishes with increasing $N$.

Despite the relative scarcity of density one packings, a statistical argument yields $\rho \rightarrow 1$ as $N \rightarrow \infty$. We show this by first generating a lower bound on the packing density for every $N$ by simply removing one or more squares from a density one packing. More precisely, suppose $M$ is either a square integer or sum of two squares. Define a new packing for $N = M-k$ by removing $k$ squares. The resulting packing density gives a lower bound of $\rho=N/(N+k)$. Given a number of squares, $N$, however, determining $k$ requires knowledge of the nearest density one packing larger than $N$. For sufficiently large $N$, we can estimate that the distance to the next density one packing is of order $\sqrt{\ln N}$ larger than $N$. Therefore, a lower bound on the packing density of $N$ squares is $1-\sqrt{\ln N}/N$ for large $N$ which yields the result the result that the density approaches one asymptotically. This simple argument does not take into account fluctuations in the spacing of sums of two squares and it would be interesting to find a mathematically rigorous asymptotic lower bound on the packing density.  For squares in a square REFERENCE shows that this bound is SQUARES IN SQUARE BOUND.

Interestingly, our numerical results suggest that a few examples of these kind of defected packings are actually optimal, as indicated in the `Comment' column in the Table using the notation $n_1^2-k$ or $n_1^2+ n_2^2-k$ depending on whether they are generated by removing $k$ squares from $N$ a square integer or a sum of two square integers respectively. Examples include $N=3$, $7$ and $15$. Note that these configurations have finite entropy since the hole(s) in the lattice can be distributed arbitrarily.

%Although for small $N$, the density of sums of two squares together with perfect squares is relatively high,  it is known RAMANUJAN ETC REFERENCE that these values of $N$ have vanishing density among all positive integers. In particular the density of sums of two squares less than $n$ behaves asymptotically as $1/\sqrt{\log(n)}$. One interesting consequence of this is that $\rho=1$ packings appear with vanishing frequency as $N  \rightarrow \infty$. 

%Unlike the perfect square packings, the sum of two squares packing must have all the rows of squares aligned and do not have finite entropy.

\subsubsection{Bricklayer packings with gaps}
Next we consider Bravais lattice solutions that have density less than one, that is $d>1$.  Because these solutions may also have  $c\neq 0$, we call these `gapped bricklayer configurations.'  An example is shown in Fig \ref{fig:gb}(A).  Eqs. (\ref{eq:gap}) allow us to numerate all gapped bricklayer configurations. Since the packing density of these configurations is $(n_2^2 + n_4^2)/N$, the highest packing density we can find within this class of configurations with density less than unity must have $n_2^2 + n_4^2 = N-1$.  This requires that $N$ be one more than a sum of two squares. The first several are 2,3,6,11,14,18,26, and 27.  Based on the numerics we believe that for $N=6$, 11, 14 and 27, the gapped bricklayer packing is optimal. These are indicated in the Table with the abbreviation `GB' in the Comment column. The associated (not unique) lattice vectors are shown in the rightmost columns of the Table for the GB packings.  There are also gapped bricklayer solution for density $N-2/N$ when $N$ is two more than a sum of two squares though we have not found any candidate optimal solutions of this form for $N \leq 27$.

Unlike the bravais lattice packings with density one, different rows of the gapped bricklayer solutions have a fixed shift given by $c = - (n_1 n_2 + n_3 n_4)/(n_2^2 + n_4^2)$. The denominator is the closest sum of two squares below $N$. This has ramifications on the packing entropy since we are no longer free to arbitrarily shift rows. In Fig. X, the lattice vectors are shown on a schematic representation of a gapped bricklayer configuration. Two rows are locked with respect to each other when a linear combination of the torus primitive vectors $\mathbf{A}_1$ and $\mathbf{A}_2$ points between the two rows. In particular, if $|n_2|$ and $|n_4|$ are mutually prime, all rows are locked and the entropy of the configuration vanishes. Alternatively, the greatest common divisor of $|n_2|$ and $|n_4|$, $g=\textrm{gcd}(|n_2|,|n_4|)$, tells us that every $g^{th}$ row is locked. Thus, the entropy of a gapped bricklayer configuration is proportional to $(n_4-n_2)(1-1/g)$.

\subsection{Non-Bravais Lattice Packings}
If the Comment section simply repeats the value of $N$ it indicates a special case for which the squares are not on a Bravais lattice and for which there is not an obvious pattern that can be extrapolated easily to optimal packings for higher values of $N$.

\subsubsection{Gapped bricklayer with domino bricks, $N=22$}
The conjectured best packing for $N=22$ is shown in Fig. \ref{fig:n12} It is actually a gapped bricklayer configuration except that the unit cell or brick is composed of two squares stacked in the $\hat{{\bf y}}$ direction (the direction perpendicular to the rows).  Except for this difference, the configuration is identical to the $N=11$ gapped bricklayer and has density $\rho= 10/11$.

\subsubsection{Lattice of $\frac{1}{2}\times\frac{1}{2}$ holes, $N=12$ and $23$}
The conjectured best configurations for $N=12$ and 23 are shown in Figs. \ref{fig:n12} and \ref{fig:n23}, respectively.  In both cases the motif can be described as a lattice of $\frac{1}{2}\times\frac{1}{2}$ holes. It is straightforward to verify that these motifs are in fact packings on the torus and have the density $N/(N+k/4)$ where $k$ is the number of holes in the unit cell.  For $N=12$, evidently $k=2$ and for $N=23$, $k=5$. 


\subsubsection{Lattice of skew squares embedded in a square lattice, $N=21$}
The configuration for  $N=21$, shown in Fig.\ \ref{fig:n21}, does not follow any of the motifs described heretofore. The configuration consists of a $4 \times 4$ square with motif of 5 squares attached to its the side.  This 5-square pattern is also the best packing of 5 squares in a square \cite{Friedman2002}.  A simple calculation yields the density, $\rho= 21/(4^2+(2+1/\sqrt{2})^2)$.  This packing has one square per unit cell tilted at 45$^{\circ}$ relative to all other squares.  This is the only example that we found for which not all of the squares in the motif are oriented in the same way. It was the most difficult configuration for our annealing algorithm to find.  Are there finitely or infinitely many other $N$ for which not all squares have the same orientation? Are there other best packings on a torus that include non-trivial packings in the square?

[ ADD IN BIT ABOUT DENSITY OF SUMS OF SQUARES].

\subsection{Entropy and rotational invariance of $\rho=1$ packings}
%JM: I prefer "Rotational invariance" to "Orientational symmetry"

%\subsubsection{Frequency of $\rho=1$ packings}

%PERHAPS THE FOLLOWING ARGUMENT SHOULD BE MOVED UP TO "PROVIDE ARGUMENT HERE" AND THIS SECTION BE DEVOTED TO ENTROPY AND ROTATIONAL SYMMETRY

%It is straightforward to show that $\rho=1$ packings are only possible for those values of $N$ such that $N$ is either a perfect square or is expressible as the sum of two squares. First, it is clear that all density one packings must be arranged in rows: every square in a $\rho=1$ packing must have at least four other squares bordering it along a finite segment length, thus forcing all $N$ squares to share the same orientation.  Then it is clear that for such oriented configurations, all of the squares must be arranged in rows, since for any three squares that touch one another and are oriented in the same direction, two of out of the three squares must define a row. Finally: if all squares for $\rho=1$ packings are thus arranged in rows, the periodicity requirement of toroidal boundary conditions ensures that the number of squares $N$ in the packing can be expressed as $N=n_1^2+n_2^2$, via the Pythagorean theorem (see Figure \ref{fig:pythagoras}).  In this way, we see that $\rho=1$ packings are only possible when $N$ is a perfect square or expressible as a sum-of-squares; and as mentioned above, this means that $\rho=1$ appear with vanishing frequency as $N  \rightarrow \infty$. 

\subsubsection{Rotational invariance of $\rho=1$ packings as $N \rightarrow \infty$}

Unlike squares packed into a square boundary, squares packed on a torus maintain rotational invariance in the thermodynamic limit.  This can be seen as follows: one can see in Figure \ref{fig:pythagoras} that any $N=n_1^2+n_2^2$ packing (which are, as seen above, the only possible packings for $\rho=1$) will orient the square lattice at an angle of $\tan^{-1}(n_2/n_1)$ relative to the torus lattice vectors.  One may approach the thermodynamic limit $N \rightarrow \infty$ by choosing a particular subsequence of integers $N_i=n_{1_i}^2+n_{2_i}^2$ such that 
%JM: I MADE THIS A STATEMENT ABOUT THE LIMIT AND REMOVED THE NOTE
$\frac{n_{1_i}}{n_{2_i}} \rightarrow r$; and the constant $r$ will thus pick out some particular orientation of the square lattice with respect to the underlying torus 
%(note that if $r$ is irrational, the ratio $\frac{n_{1_i}}{n_{2_i}}$ approaches, but never equals, $r$). 
Thus the thermodynamic limit of $\rho=1$ packings on the torus  preserves orientational symmetry.

\subsubsection{Entropy of $\rho=1$ packings}

%JM: I DID SOME EDITING OF THE FOLLOWING PARAGRAPH
The entropy of $\rho=1$ packings depends on $N$ in the following way.  Because $\rho=1$ packings must always arranged in rows (see above), the only contribution to the entropy comes from the freedom
of row(s) to shift relative to the each other.  Thus the entropy will be proportional to the number of rows that can be  shifted independently.  In Figure \ref{fig:pythagoras} we see that the number of rows that can be shifted independently is proportional to the smallest integer that can be written in the form $d=a n_2 + b n_4$, where $n_2 and n_4$ are defined as above (and in Figure \ref{fig:Bezout}), and $a$ and $b$ are integers.  Bezout's Lemma [REF] states that this integer linear combination can be made equal to, but not smaller than, $d$, where $d$ is the greatest common divisor of $n_2$ and $n_4$;  thus, the entropy of $\rho=1$ packings will be proportional to $d$.  

\subsection{Table of Results}

To summarize: the perfect square, sum of square and gapped bricklayer configurations cover most of the case we have found for $N \leq 27$.  The Table gives exact configurations, if $\rho=1$, or conjectured configurations, if $\rho<1$, for each value of  $N$ less than 28.  The column `$\rho$' is the density of the configuration. The `Comment' column describes the type of lattice.   For example, $3^2$ indicates a perfect square and $5^2-1$ indicates a perfect square with one square missing.  Similarly $3^2+1^2$ refers to the sum of two squares and `GB' stands for `gapped bricklayer.'  If a single number is in the `Comment' column it refers to one of the special cases discussed above.  The four columns `$n_1$, $n_2$, $n_3$, $n_4$' are shown if the configuration of squares is itself a Bravais lattice and these integers are the coefficients of the lattice vectors of the torus in terms of the lattice vectors of the squares as defined in (\ref{eqn:Ana}).  The minimum number of columns needed to specify the lattice are filled in.  Finally, for the gapped bricklayer configurations, the shift $m$, as discussed above is given.

\vspace{.3in}
\begin{table}
\label{table}
\caption{Exact and conjectured closest packed configurations of squares on a torus.  Refer to the text for the meaning of the columns.}
  \begin{tabular}{|c | c|c | c| c | r | c | c | c |}
\hline
% after \\ : \hline or \cline{col1-col2} \cline{col3-col4} ...
  $N$ &$\rho$&Comment& $n_1$ & $n_2$ & $n_3$ & $n_4$  \\ \hline \hline
   1 &1&  $1^2$ & 1&  &   &    \\ \hline 
   2 &1&  $1^2+1^2$ &1& 1 &  &  \\ \hline
   3 &$3/4=0.75$&  $2^2 -1$ &2&  & &   \\ \hline
   4 &1&  $2^2$ &2& & &   \\ \hline
   5 &1&  $2^2+1^2$ &2& 1 &    & \\ \hline
   6 &$5/6=0.8\bar{3}$& GB &2& -1 &2 & 2   \\ \hline
   7 &$7/8=0.875$& $2^2+2^2 -1$ &2& 2 & &     \\ \hline
   8 &1& $2^2+2^2 $ &2& 2 & &     \\ \hline
   9 &1&  $3^2$ &3& & &  \\ \hline
   10 &1&  $3^2+1^2$ &3& 1 &    & \\ \hline
   11 &$10/11=0.\overline{90}$& GB &3& 1 &-2 & 3   \\ \hline
   12 &$24/25=0.96$& 12 && & &    \\ \hline
   13 &1&  $3^2+2^2$ &3& 2 &    & \\ \hline
   14 &$13/14=0.9\overline{285714}$& GB &4& 3 &-2 & 2   \\ \hline
   15 &$15/16=0.9375$&  $4^2 -1$ &4&  & &   \\ \hline
   16 &1&  $4^2$ &3& & &   \\ \hline
   17 &1&  $4^2+1^2$ &4& 1 &    & \\ \hline
   18 &1&  $3^2+3^2$ &3& 3 &    & \\ \hline
   19 &$19/20=0.95$&  $4^2 +2^2 -1$ &4& 2 & &   \\ \hline
   20 &1&  $4^2+2^2$ &4& 2 &    & \\ \hline
   21 &$21/(4^2+(2+1/\sqrt{2})^2)=0.900189 \ldots$&  $21$ & &  &   &    \\ \hline 
   22 &$10/11=0.\overline{90}$& 22 &3& 1 &-2 & 3  \\ \hline
   23 &$92/97=0.94845 \ldots$&  $23$ & &  &   &    \\ \hline
   24 &$24/25=0.96$&  $5^2 -1$ &5&  & &   \\ \hline
   25 &$1$&  $5^2$ &5&  & &   \\ \hline
   26 &1&  $4^2+3^2$ &4& 3 &    & \\ \hline
   27 &$26/27=0.\overline{962}$& GB &2&5 &-5 & 3  \\ \hline

\hline
\end{tabular}
\end{table}




