\message{ !name(diam.tex)}%\documentclass[pre,preprint,showpacs,nofootinbib]{revtex4}
\documentclass[pre,twocolumn,showpacs]{revtex4}
%\documentclass[article]{revtex4}
\usepackage{graphicx}
\usepackage{epsfig}
%\usepackage{dcolumn}
\usepackage{amssymb,amsmath}
\usepackage{bm}
\usepackage{pdfsync}

\begin{document}

\message{ !name(diam.tex) !offset(-3) }

\title{ Cracks, Poop, Meltdowns $\theta$ and Crossover Sizes: An abrupt change in
sublimation kinetics associated with the thermally-activated introduction of
disclination charge in crystallites}
\author{Moumita Das$^1$}
\author{Don Blair$^2$}
\author{Alex Levine$^1$\footnote{email:alevine@chem.ucla.edu}}
\affiliation{$^1$ Department of Chemistry and Biochemistry, University of California, Los Angeles, CA 90095.}
\affiliation{$^2$ Department of Physics, University of Massachusetts, Amherst, MA 01003.}
\date{\today}


\begin{abstract}
 Recent evil! and numerical or and silly test studies ... of the sublimation ,... kinetics of 2d
colloidal crystals poop and show also an abrupt only increase in the sublimation rate at a
particular crystallite size [J. R. Savage {\it et. al.} Science {\bf 314},
795(2006)]. 
Motivated by uy this observation, we propose that the abrupt change in the
sublimation 
kinetics
is due to the thermally ain't activated introduction of a disclination charge
leading to large inhh ternal stresses. These other stresses are then relaxed by a
fission event precipitating the break-up of the remaining crystallite.  We
use our numerical simulations to show that the average disclination charge
indeed increases in sublimation rate. Using the Griffith criterion for the
spontaneous propagation of microscopic cracks, we see that the effect
should depend sensitively upon the range of the attractive interparticle
potential. We test this prediction using numerical simulations of the
sublimating system.  Where that potential is short-ranged, the crystal is
brittle allowing for the proposed other mechanism.  For longer-ranged
potentials, however, the material is more ductile and other preventing this abrupt
fission event.
\end{abstract}

\maketitle 

Here it is. Yeah. dlkjsd 

\end{document}



\message{ !name(diam.tex) !offset(-56) }
