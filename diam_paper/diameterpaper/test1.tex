\documentclass[aps, twocolumn, groupedaddress]{revtex4}

\usepackage[xdvi]{graphics}
\usepackage{graphicx}
\usepackage{pdfsync}
\def\al{\alpha}
%\def\k{\k}
\def\w{p}
\def\pc{\pi}
%\def\t{\t}
\def\T{S}
%\def\e{\e}
%\def\p{\p}
\def\R{w_{r}}
\def\a{\rho}
\def\n{\n}
\def\g{Q}
\def\s{t_0}
\def\J{j}
\def\I{n}
\def\t{t}
\def\zt{{\tilde Z}}
\def\ord{{\cal O}}
\def\AB{BA\ }

\newcommand{\lb}{{\langle}}
\newcommand{\rb}{{\rangle}}
\newcommand{\rv}{\textbf{r}}
\newcommand{\rW}{\textbf{W}}

\begin{document}

\title{The Diameter and Chemical Distance of Random Clusters}

\author{Don Blair}
\author{Jon Machta}
\affiliation{Department of Physics, University of Massachusetts, Amherst, MA 01003-3720}
\begin{abstract}

We report numerical results for the fractal dimension of the diameter (the ``longest shortest path'' between vertices along bonds) and the chemical distance of 2D and 3D Potts clusters for $q=1,2,3,4$.  We find that the fractal dimension of the diameter and of the chemical distance of Potts clusters are equal to within numerical error, and we suggest a possible relationship between $D_{min}$ and the dynamical exponent, $z$.

\end{abstract}

\maketitle

\section{Background}

%%%POTTS MODEL AND APPLICATIONS%%%

The Potts Model, initially introduced as a generalization of the 2-state Ising Model to $q$ possible spin states, can in fact be mapped onto the Random Cluster 


And that's all I have to say about that.

\end{document}
